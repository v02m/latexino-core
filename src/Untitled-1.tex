% ===========================
% 🖥️ ЛИСТИНГИ И ТЕСТОВЫЙ ТЕКСТ
% ===========================

\ProvidesPackage{listingssetup}[2025/04/21 Listings and placeholder text setup]

% Пакет для вставки исходного кода
\RequirePackage{listings}

% Вставка "слепого" текста (для тестирования компоновки)
\RequirePackage{blindtext}  % \Blinddocument, \blindtext
\RequirePackage{lipsum}     % \lipsum[1-5] и др.

% ======================================================================
% 🖥️ Стиль листингов (ГОСТ-дружелюбный, читаемый, без излишеств)
% ======================================================================
\lstset{
  basicstyle=\ttfamily\small,        % Моноширинный, мелкий шрифт
  breaklines=true,                   % Автоматический перенос длинных строк
  breakatwhitespace=true,            % Перенос только по пробелам
  frame=single,                      % Рамка вокруг кода
  rulecolor=\color{black},           % Чёрная рамка (ГОСТ: без цвета)
  backgroundcolor=\color{gray!5},    % Очень лёгкий фон (не отвлекает)
  keywordstyle=\bfseries,            % Ключевые слова — жирные (без цвета!)
  commentstyle=\itshape,             % Комментарии — курсив
  stringstyle=\ttfamily,             % Строки — моноширинные
  numbers=left,                      % Нумерация слева
  numberstyle=\tiny\color{gray},     % Мелкие серые номера
  tabsize=2,                         % Размер табуляции
  showstringspaces=false,            % Не показывать пробелы в строках
  captionpos=t                       % Подпись сверху (как у таблиц!)
}
% Примечание:
% Все эти пакеты помогают отлаживать структуру документа, демонстрировать примеры,
% и использовать реальные куски кода.
