\documentclass[12pt]{memoir}

% ======================================================================
% Подключение центрального пакета xememoir
% ======================================================================
\usepackage{xememoir}

% ======================================================================
% Настройка флагов (типографика, графика, списки)
% ======================================================================
% Типографские профили
%\@xem@mstrue
%\@xem@classictrue
%\@xem@modernfalse
%\@xem@academicfalse
%\@xem@poetryfalse
%\@xem@technicalfalse
%\@xem@elegantfalse
%\@xem@compactfalse
%\@xem@boldfalse
%
%% Графические профили
%\@xemg@classictrue
%\@xemg@colorfulfalse
%\@xemg@minimalfalse
%\@xemg@technicalfalse
%\@xemg@presentationfalse
%
%% Стили списков
%\@xeml@gosttrue
%\@xeml@numericfalse
%\@xeml@mixedfalse
%\@xeml@alphafalse

% ======================================================================
% Заголовок документа
% ======================================================================
\title{Полный тест пакета xememoir}
\author{Valery}
\date{\today}

\begin{document}

	\maketitle

	% ======================================================================
	% Проверка шрифтов
	% ======================================================================
	\chapter{Тест шрифтов}
	Активные шрифты:
	\makeatletter
	MS: \iffontfamily@ms TRUE\else FALSE\fi,
	Liberation: \iffontfamily@liberation TRUE\else FALSE\fi,
	CMU: \iffontfamily@cmu  TRUE\else FALSE\fi.
	\makeatother
	\textbf{Bold text} и \textit{italic text} для проверки стилей.

	% ======================================================================
	% Проверка полей и layout
	% ======================================================================
	\chapter{Тест полей и колонтитулов}
	Здесь мы проверяем корректность установки полей, колонтитулов и заголовков.
	%\lipsum[1-2] % Если нужен пример текста для проверки макета

	% ======================================================================
	% Проверка списков
	% ======================================================================
	\chapter{Тест списков}
	\section{Элементы по ГОСТ}
	\begin{itemize}
		\item Первый элемент
		\item Второй элемент
		\item Третий элемент
	\end{itemize}

	\section{Нумерованный список}
	\begin{enumerate}
		\item Первый
		\item Второй
		\item Третий
	\end{enumerate}

	% ======================================================================
	% Проверка графики
	% ======================================================================
	\chapter{Тест графических профилей}
%	Classic: \ifgraphicstyle@classic  TRUE\else FALSE\fi,
%	Colorful: \ifgraphicstyle@colorful   TRUE\else FALSE\fi,
%	Presentation: \ifgraphicstyle@presentation TRUE\else FALSE\fi.
%
%	% Пример вставки графики
%	\IfFileExists{example-image-a.pdf}{%
%		\includegraphics[width=0.5\textwidth]{example-image-a.pdf}%
%	}{%
%		\typeout{No example image found. Skipping.}%
%	}

	% ======================================================================
	% Проверка модулей math, tables, bib, todo, listings, pdf
	% ======================================================================
	\chapter{Дополнительные модули}

	\section{Math}
	Пример формулы:
	\[
	E = mc^2
	\]

	\section{Tables}
	\begin{tabular}{|c|c|c|}
		\hline
		A & B & C \\
		\hline
		1 & 2 & 3 \\
		\hline
	\end{tabular}

	\section{Bibliography}
	% Простейший тест
%	\cite{dummyref} % Если bibsetup подключён корректно

	\section{TODO notes}
	% Проверка todo-заметок
%	\todo{Это тестовая заметка TODO.}
%
%	\section{Listings / code}
%	\begin{lstlisting}[language=Python]
%		def test():
%		return "Hello, xememoir!"
%	\end{lstlisting}

	\section{PDF / гиперссылки}
	Проверка гиперссылки: \href{https://www.ctan.org}{CTAN}

	% ======================================================================
	% Заключение
	% ======================================================================
	\chapter{Заключение}
	Если все флаги и модули активны и выводятся корректно, значит central module работает правильно.

\end{document}

