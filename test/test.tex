\documentclass[12pt]{memoir}

% ======================================================================
% Подключение центрального пакета xememoir
% ======================================================================
\usepackage{ifthen} % для демонстрации условной логики, если потребуется
\usepackage{xememoir}

% ======================================================================
% Настройка флагов вручную (для теста)
% ======================================================================
% Типографские профили
\@xem@mstrue
\@xem@classictrue
\@xem@modernfalse
\@xem@academicfalse
\@xem@poetryfalse
\@xem@technicalfalse
\@xem@elegantfalse
\@xem@compactfalse
\@xem@boldfalse

% Графические профили
\@xemg@classictrue
\@xemg@colorfulfalse
\@xemg@minimalfalse
\@xemg@technicalfalse
\@xemg@presentationfalse

% Стили списков
\@xeml@gosttrue
\@xeml@numericfalse
\@xeml@mixedfalse
\@xeml@alphafalse

% ======================================================================
% Заголовок документа
% ======================================================================
\title{Тестовый документ для пакета xememoir}
\author{Valery}
\date{\today}

\begin{document}

	\maketitle

	% ======================================================================
	% Тест полей, шрифтов, списков и гиперссылок
	% ======================================================================

	\chapter{Введение}
	Это тестовый документ для проверки работы центрального модуля \texttt{xememoir.sty}.
	Ниже пример списка:

	\begin{itemize}
		\item Первый элемент
		\item Второй элемент
		\item Третий элемент
	\end{itemize}

	\section{Тест гиперссылок}
	Проверим ссылку на \href{https://www.ctan.org}{CTAN}.

	\section{Тест шрифтов и типов профилей}
	Проверим активные флаги и шрифты.
	\textbf{Bold}: активность флага \if@xem@bold TRUE\else FALSE\fi.
	\textit{Classic profile}: \if@xem@classic TRUE\else FALSE\fi.
	MS font: \if@xem@ms TRUE\else FALSE\fi.

	\section{Тест графики}
	Проверка графических профилей: Classic=\if@xemg@classic TRUE\else FALSE\fi, Colorful=\if@xemg@colorful TRUE\else FALSE\fi.

	\section{Заключение}
	Если диагностика пакета в лог-файле показывает корректные флаги, значит центральный модуль и подключение всех модулей работают правильно.

\end{document}

